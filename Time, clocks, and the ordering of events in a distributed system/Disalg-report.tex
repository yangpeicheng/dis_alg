\documentclass[UTF8]{article}

%--
\usepackage{ctex}
\usepackage[margin=1in]{geometry}
\usepackage{amsmath}
\usepackage{amssymb}



%--
\begin{document}
    
%--
{\flushleft \bf \Large 姓名:} 杨佩成

{\flushleft \bf \Large 学号:} MG1733079

{\flushleft \bf \Large 日期:} 2017.12.10


%=========================================================================
\section*{论文信息}
    
L. Lamport, Time, clocks, and the ordering of events in a distributed system. Communications of the ACM. 21, 558–565 (1978).


%=========================================================================
\section{概述}

	分布式系统是一组不同进程的集合,这些进程在空间上是分离的并且通过交换消息来进行通信。在分布式系统中,有时很难确定两个事件中哪个先发生。“happen before”关系只是整个系统中的一个偏序。这篇文章讨论了用“happen before”定义的偏序,并

\section{偏序}
  
	

	基本假设:
	
	系统由一组进程组成
	
	单个进程是一组有序事件的集合,发送消息和收到消息也是事件

	定义“happened before”关系,用“$\to$”表示“happened before”关系。

	定义:(1)若$a$和$b$是用一个进程中的两个事件,并且$a$在$b$之前发生,则$a \to b$。

		(2)若$a$和$b$在两个不同的进程中,$a$是一条消息的发送方,$b$是同一条消息的接收方,则$a \to b$。

		(3)若$a \to b$,$b \to c$,则$a \to c$。若$a \nrightarrow b$,$b \nrightarrow a$,则称事件$a$和$b$是同步的。

	逻辑时钟
	
	我们通过给每个事件分配一个数字来给系统引入时钟。给每一个进程$P_i$定义一个时钟$C_i$,$C_i$实际上是一个函数,$C_i \langle a \rangle$表示进程$P_i$给事件$a$分配的数字。整个系统的时钟用函数$C$表示,系统给事件$b$分配的数字表示为$C \langle b \rangle$。若$b$是进程$P_j$中的事件,则$C\langle b\rangle =C_j\langle b\rangle$。
	
	时钟条件:
	
	对于任意事件$a$,$b$:若$a \to b$,则$C \langle a \rangle < C \langle b \rangle$。
	
	只要下面两个条件满足,时钟条件就可以满足
	
	$C1$.若$a$和$b$是进程$P_i$的两个事件,$a$在$b$之前发生,则$C_i \langle a \rangle < C_i \langle b \rangle$。

	$C2$.若$a$在进程$P_i$中,$b$在进程$P_j$中,$a$是一条消息的发送方,$b$是同一条消息的接收方,则$C_i \langle a \rangle < C_j \langle b \rangle$。

	如果要在实际系统中引入时钟并且能够满足时钟条件,只需要满足以下实现规则:

	$IR1$.每个进程$P_i$维护一个$C_i$,并且当有新的时间发生时,$C_i$加一。

	$IR2$.$(a)$进程$P_i$中的事件$a$在发送信息$m$时,会包含一个时间戳$T_m$,$T_m$等于$C_i \langle a \rangle$。
		 $(b)$当进程$P_j$收到消息$m$后,$P_j$会更新$C_j$,保证$C_j$大于等于当前值并且大于$T_m$。

	规则$IR1$保证了系统满足$C1$,$IR2$保证了满足$C2$。因此只要系统满足了上述的规则,就可以保证系统有一个正确的逻辑时钟。

	在系统中引入时钟,我们就可以对系统中的所有事件进行排序(只需要按照它们发生的时间进行排序)。为了打破系统中可能的同步关系,我们在进程之间引入任意的顺序关系$\prec$,这样我们就可以定义系统的全序关系$\Rightarrow$:
	
	$a$是进程$P_i$中的一个事件,$b$是进程$P_j$中的一个事件,若$(\romannumeral1)C_i\langle a \rangle < C_j \langle b \rangle$或$(\romannumeral2) C_i\langle a \rangle = C_j \langle b \rangle$并且$P_i \prec P_j$,则$a \Rightarrow b$。

	从全序关系的定义和时钟条件可以证明如果$a \rightarrow b$,则$a \Rightarrow b$。全序关系依赖于系统的时钟,并且系统的全序关系是不唯一的。不同的时钟,只要满足了时钟条件,就可以生成不同全序。在分布式系统中,只有偏序是唯一的。

	在分布式系统中引入逻辑时钟,可以得到一个系统的全序。通过全序,我们可以解决一类互斥问题。

	基本问题:

	在一个分布式系统中,有一组固定的进程共享一个资源。在每个时刻只能有一个进程能够使用这个资源。分配资源的算法必须满足下面三点要求:(1)资源在被分配给新的进程之前必须先被旧的进程释放。(2)资源的分配顺序必须和进程的请求顺序一致。(3)每个获取了资源的进程最终都会释放资源,每个请求最终都会得到授权。
	
	使用一个集中调度的进程来处理授权在一些情况下会失效。假设$P_0$是调度进程,$P_1$先给$P_0$发送一个资源授权请求,再给$P_2$发送一个消息。$P_2$在收到消息后也发送一个请求给$P_0$。如果$P_2$的请求先到达,那么上面的条件3就会不满足。
	
	我们可以在系统中引入逻辑时钟来解决这个问题。为了简化问题,我们假设两个进程之间消息的到达顺序与发送顺序一致,每个消息最终都能被收到,每个进程都能与其他任一进程通信。每个进程都维护一个仅自己可见的请求队列,初始情况下队列只有一条消息$T_0:P_0$请求资源。$P_0$是最初被授予资源的进程,$T_0$小于任意时钟初始值。具体算法由下面五条规则定义:

	1.当进程$P_i$请求资源时,它会向其它进程广播$T_m:P_i$请求资源的消息并把消息加入自己的请求队列中,其中$T_m$是发送消息的时间戳。

	2.当进程$P_j$收到$T_m:P_i$请求资源的消息时,它会将消息放入自己的请求队列中并回复一个带时间戳的$ack$给$P_i$。

	3.若$P_i$要释放资源,它会将$T_m:P_i$请求资源的消息从自己的请求队列中删除,并向其它进程广播一个带时间戳的释放资源的消息。

	4.当进程$P_j$收到来自$P_i$的释放资源消息时,它会从请求队列中将$T_m:P_i$请求资源的消息删除。
	
	5.当下面两个条件满足时,进程$P_i$会被授予资源:$\romannumeral1$在它自己的消息队列中,不存在比$T_m:P_i$更早的请求资源消息,这里的顺序通过全序定义的。$romannumeral2$$P_i$发送的请求收到其它进程的回复。

	这是一个分布式的算法,不需要中心节点,每个进程都独立地运行。这种算法可以实现分布式系统中的任何同步操作。因为所有的进程都是按照时间戳(由系统的全序决定)来执行命令的,每个进程都是执行的一个相同的命令序列。但是这种算法需要所有的进程都积极参与,当一个进程发生故障时,其它进程就不能继续执行命令,这会导致整个系统的停止。如果没有物理时钟,我们是无法区分进程故障和进程暂停的。

	异常动作:

	我们的资源调度算法是通过系统的全序来对请求排序,这就存在发生异常行为的可能。假设一个人在计算机A上提出一个请求A,然后打电话给一个朋友让他在电脑B上提出一个请求B,系统很有可能先对B进行响应。系统并不知道A在B之前发生,因为这个先后顺序信息是基于系统外部的消息的。假设$\varphi$是系统内部所有事件的集合,\underline{$\varphi$}是系统内部事件和与内部事件相关的外部事件的并集。我们用$\boldsymbol{\rightarrow}$表示\underline{$\varphi$}中的“happened before”关系。$A \boldsymbol{\rightarrow} B$不能推出$A \rightarrow B$。因为我们不能保证系统的内部事件能推出A在B之前发生。

%--
\end{document}