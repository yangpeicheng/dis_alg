\documentclass[UTF8]{article}

%--
\usepackage{ctex}
\usepackage[margin=1in]{geometry}
\usepackage{amsmath}
\usepackage{amssymb}



%--
\begin{document}
    
%--
{\flushleft \bf \Large 姓名:} 杨佩成

{\flushleft \bf \Large 学号:} MG1733079

{\flushleft \bf \Large 日期:} 2017.12.10


%=========================================================================
\section*{论文信息}
    
L. Lamport, Time, clocks, and the ordering of events in a distributed system. Communications of the ACM. 21, 558–565 (1978).


%=========================================================================
\section{概述}

	分布式系统是一组不同进程的集合,这些进程在空间上是分离的并且通过交换消息来进行通信。在分布式系统中,有时很难确定两个事件中哪个先发生。“happen before”关系只是整个系统中的一个偏序。这篇文章讨论了用“happen before”定义的偏序,并
    
	偏序

	基本假设:
	
	系统由一组进程组成
	
	单个进程是一组有序事件的集合,发送消息和收到消息也是事件

	定义“happen before”关系,用“$\to$”表示“happen before”关系。

	定义:(1)若$a$和$b$是用一个进程中的两个事件,并且$a$在$b$之前发生,则$a \to b$。

		(2)若$a$和$b$在两个不同的进程中,$a$是一条消息的发送方,$b$是同一条消息的接收方,则$a \to b$。

		(3)若$a \to b$,$b \to c$,则$a \to c$。若$a \nrightarrow b$,$b \nrightarrow a$,则称事件$a$和$b$是同步的。

	逻辑时钟
	
	我们通过给每个事件分配一个数字来给系统引入时钟。给每一个进程$P_i$定义一个时钟$C_i$,$C_i$实际上是一个函数,$C_i \langle a \rangle$表示进程$P_i$给事件$a$分配的数字。整个系统的时钟用函数$C$表示,系统给事件$b$分配的数字表示为$C \langle b \rangle$。若$b$是进程$P_j$中的事件,则$C\langle b\rangle =C_j\langle b\rangle$。
	
	时钟条件:
	
	对于任意事件$a$,$b$:若$a \to b$,则$C \langle a \rangle < C \langle b \rangle$。
	
	只要下面两个条件满足,时钟条件就可以满足
	
	若$a$和$b$是进程$P_i$的两个事件,$a$在$b$之前发生,则$C_i \langle a \rangle < C_i \langle b \rangle$。

	若$a$在进程$P_i$中,$b$在进程$P_j$中,$a$是一条消息的发送方,$b$是同一条消息的接收方,则$C_i \langle a \rangle < C_j \langle b \rangle$。
%--
\end{document}