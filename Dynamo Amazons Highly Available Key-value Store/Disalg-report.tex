\documentclass[UTF8]{article}

%--
\usepackage{ctex}
\usepackage[margin=1in]{geometry}

%--
\begin{document}
    
%--
{\flushleft \bf \Large 姓名:} 杨佩成

{\flushleft \bf \Large 学号:} MG1733079

{\flushleft \bf \Large 日期:} 2018.1.12


%=========================================================================
\section*{论文信息}
    
Decandia G, Hastorun D, Jampani M, et al. Dynamo: amazon's highly available key-value store[C]// ACM Sigops Symposium on Operating Systems Principles. ACM, 2007:205-220.
    
%=========================================================================
\section{章节1}
	Dynamo是Amazon设计的高可用性的键值存储系统。Amazon使用了一个高度分散、松耦合和面向服务的架构,在这种环境下,存储系统需要一直保持可用。Amazon的基础设施由上百万的组件构成,服务器错误和网络组件错误是经常出现的,Dynamo需要在保证系统可用性和性能的同时应对各种错误的发生。Dynamo还需要有高度可扩展性,通过增加节点实现容量和性能的线性扩展。除此之外,Dynamo的设计还需要满足以下要求和假设:
\begin{itemize}
	\item
		查询模型:一条数据可以用一个唯一的Key来确定,没有跨越多条数据的操作和关系模型。Dynamo所面向应用存储的数据很小,通常小于1MB。
	\item
           ACID属性:ACID是一组保证数据库正确运行的属性,但是保证ACID会损失系统的可用性。Dynamo弱化了一致性的要求而希望提高可用性。Dynamo不提供隔离性保证,只允许单条数据的更新。
	\item
           效率:系统的延迟要满足SLA(Service Level Agreement),服务可以根据自己的要求配置Dynamo。
	\item
           其它假设:Dynamo只在Amazon内部使用,所以没有安全上的要求。每个服务运行一个Dynamo的实例,要保证系统的可扩展性。
		
\end{itemize}•
%--    
\begin{itemize}
    \item 基本模型、基本假设:分布式系统的基本模型往往比较复杂。实际论文中的模型与课本中的模型比,要精细地多。大家需要按照课上讲的框架,把论文中模型的各个纬度都讨论清楚;
    
    \item 基本问题:是否是经典问题。即使对于经典问题(e.g. consensus),也有很多的变体。把问题全面认识清楚,与解决问题一样重要。对于一些不经典的问题,可以参照课上的学习,将问题全面地认识清楚,解释清楚;
    
    \item 主要贡献:对于论文贡献的深入解释,往往是一个具体的模型、算法、分析技术。技术的细节不是最重要的,简单copy原文更是大忌。课程看重的是,从课程中所学习的视角进行分析解读,进行原理性的提炼、阐述等。另外,虽然我们是偏理论的课程,但是对于实现、实验、系统的解读也是欢迎的,例如,系统设计、实现的背后,理论建模与分析是如何发挥指导性作用的,等等。
    
     
\end{itemize}


%=========================================================================
\section{章节2}

。。。
    
%--
\end{document}